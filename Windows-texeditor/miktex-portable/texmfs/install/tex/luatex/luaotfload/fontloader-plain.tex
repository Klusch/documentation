%D \module
%D   [       file=luatex-plain,
%D        version=2009.12.01,
%D          title=\LUATEX\ Macros,
%D       subtitle=Plain Format,
%D         author=Hans Hagen,
%D           date=\currentdate,
%D      copyright={PRAGMA ADE \& \CONTEXT\ Development Team}]

\input plain

\directlua {tex.enableprimitives('', tex.extraprimitives())}

% We assume that pdf is used.

\ifdefined\pdfextension
    \input luatex-pdf \relax
\fi

\outputmode 1

% \outputmode 0 \magnification\magstep5

% We set the page dimensions because otherwise the backend does weird things
% when we have for instance this on a line of its own:
%
%   \hbox to 100cm {\hss wide indeed\hss}
%
% The page dimension calculation is a fuzzy one as there are some compensations
% for the \hoffset and \voffset and such. I remember long discussions and much
% trial and error in figuring this out during pdftex development times. Where
% a dvi driver will project on a papersize (and thereby clip) the pdf backend
% has to deal with the lack of a page concept on tex by some guessing. Normally
% a macro package will set the dimensions to something reasonable anyway.

\pagewidth   8.5truein
\pageheight 11.0truein

% We load some code at runtime:

\everyjob \expandafter {%
    \the\everyjob
    \input {luatex-basics}%
    \input {luatex-fonts}%
    \input {luatex-math}%
    \input {luatex-languages}%
    \input {luatex-mplib}%
    \input {luatex-gadgets}%
}

% We also patch the version number:

\edef\fmtversion{\fmtversion+luatex}

\dump
